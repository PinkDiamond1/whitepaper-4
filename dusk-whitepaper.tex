\documentclass[sigconf]{acmart}
\usepackage[utf8]{inputenc}

\usepackage{booktabs} % For formal tables
\usepackage{enumitem}
\usepackage{csquotes}
\usepackage{amsmath}
\usepackage{algorithm}
\usepackage[noend]{algpseudocode}
\usepackage{varwidth}

    \makeatletter
    \renewcommand\subsubsection{\@startsection{subsubsection}{3}{\z@}%
                              {-3.25ex\@plus -1ex \@minus -.2ex}%
                              {.3ex \@plus .2ex}%
                              {\normalfont\large\bfseries}}% from \normalsize
    \renewcommand\paragraph{\@startsection{paragraph}{4}{\z@}%
                              {-1.8ex\@plus -1ex \@minus -.2ex}%
                              {.3ex \@plus .2ex}%
                              {\normalfont\large\bfseries}}% from \normalsize
    \def\BState{\State\hskip-\ALG@thistlm}
    \newcommand{\algorithmicbreak}{\textbf{break}}
    \newcommand{\Break}{\State \algorithmicbreak}
    \newcommand\CONDITION[2]%
    {\begin{tabular}[t]{@{}l@{}l@{}}
        #1&#2
    \end{tabular}%
    }
    \algdef{SE}[IF]{If}{EndIf}[1]%
        {\algorithmicif\ \CONDITION{#1}{\ \algorithmicthen}}%
        {\algorithmicend\ \algorithmicif}%
    \makeatother

% Copyright
\setcopyright{none}
\setcopyright{iw3c2w3}




\begin{document}
\title{The Dusk Network And Blockchain Architecture}

\subtitle{Scalable consensus and low-latency data transmissions for privacy-driven cryptosystems}

\author{Emanuele Francioni}
\affiliation{%
  \institution{Dusk Foundation}
  \streetaddress{Van Eeghenstraat 94}
  \city{Amsterdam}
  \state{The Netherlands}
  \postcode{1071 GL}}
\email{emanuele@Dusk.network}

\author{Fulvio Venturelli}
\affiliation{%
  \institution{Dusk Foundation}
  \streetaddress{Van Eeghenstraat 94}
  \city{Amsterdam}
  \state{The Netherlands}
  \postcode{1071 GL}}
\email{fulvio@Dusk.network}

% The default list of authors is too long for headers.
\renewcommand{\shortauthors}{E. Francioni and F. Venturelli}

\begin{abstract}
In order to satisfy a broad set of data transfer scenario, the \textrm{Dusk} network adds an additional layer of security to the IP protocol suite (used mostly in a peer-to-peer fashion). Through the adoption of a mix of established strategies and novel techniques, the \textrm{Dusk} network has been conceived specifically to protect the privacy of the communicating peers from any form of eavesdropping while satisfying a variety of challenging use cases varying from fast communication (e.g. voice calls) to large data transfer (e.g. file transmission). \textrm{Dusk} circumvents the notorious unreliability of crowd-sourced infrastructures by embedding economic incentives into the core mechanism of the network itself. Such incentives are designed to encourage peers to partake in the network in a permission-less, anonymous and private fashion.

\end{abstract}

\keywords{Dusk, blockchain, cryptocurrency, privacy, consensus, segregated byzantine agreement}

\settopmatter{printfolios=true}
\maketitle

\input{Dusk-preliminary}
\input{Dusk-sba}
\input{Dusk-network}
\section{Conclusions}

\textrm{Dusk} Network is an unrestricted, unsurveilled and fully distributed cryptosystem designed for high-rate voice and data communications, enforcing the utmost level of privacy to the partaking peers. The network features a novel blockchain-based digital cash called \textrm{Dusk}, used to directly incentivize participation in the network and promote widespread adoption. \textrm{Dusk} features untraceability through the use of ring confidential transactions, unlinkability through the use of stealth address and protection from Sybil attack and double spending through a novel consensus algorithm called Segregated Byzantine Agreement or SBA$\large\star$. SBA$\large\star$ provides direct block finality by preventing forking while providing virtually unbounded scalability. The network is built on top of an efficient gossip network which utilizes non-repliable datagram and garlic routing in order to prevent IP Address propagation. Finally, the \textrm{Dusk} Network is complemented with an off- online file transfer mechanism and with realtime \textrm{Dusk} payment channel to enable undetectable and fast peer-to-peer data communication through a technique we call Secure Tunnel Switching.

\bibliographystyle{ACM-Reference-Format}
\bibliography{bibliography}

\end{document}
